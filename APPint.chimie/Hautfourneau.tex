\documentclass[11pt,a4paper]{report}
\usepackage[utf8]{inputenc}
\usepackage[francais]{babel}
\usepackage[T1]{fontenc}
\usepackage{amsmath}
\usepackage{amsfonts}
\usepackage{graphicx}
\usepackage{amssymb}
\usepackage{fullpage}
\usepackage{fancybox}
\usepackage[Lenny]{fncychap}
\usepackage{gensymb}
\usepackage{color}
\usepackage{array}
\usepackage{url}
\usepackage{here}
\usepackage[final]{pdfpages}
\usepackage{wrapfig}
\usepackage{etex}%chimie
\usepackage{m-pictex,m-ch-en}%chimie
\author{Groupe 118.5}
\title{APP: Haut-fourneau}
\date{\today}
%\thispagestyle{empty}
%\setcounter{page}{0}
\begin{document}

\maketitle

\chapter{Le Haut-fourneau}
\textbf{Quantités de départ:}\\

1 tonne de minerai de fer composé de:
\begin{itemize}
\item{38\% de \chemical{Fe_2O_3}\chemical{->} 2375 moles}
\item{14\% de \chemical{FeO}\chemical{->} 1944,44 moles}
\item{24\% de \chemical{Si}\chemical{->}  4000 moles}
\item{10\% de \chemical{CaCO_3}\chemical{->} 1000 moles}
\item{4\% de \chemical{Al_2O_3}\chemical{->} 342,16 moles}
\item{10\% de \chemical{H_2O}\chemical{->} 5555,55 moles}\\
\end{itemize}


Une certaine quantité de coke (à déterminer):
\begin{itemize}
\item{90\% de \chemical{C}}
\item{10\% d'impuretés qui n'entre pas en compte dans nos équations}
\end{itemize}
On sait que 52,5\% de masse de cette coke va être utilisé exclusivement à fournir de la chaleur.\\

\textbf{Quantités à l'arrivé:}\\

Fonte:
\begin{itemize}
\item{92,9\% de \chemical{Fe}}
\item{6,1\% de \chemical{C}}
\item{1\% de \chemical{Si}}\\
\end{itemize}

Laitier:
\begin{itemize}
\item{0,6\% de \chemical{FeSiO_3}}
\item{99,4\% de \chemical{CaAl_2O_4}, \chemical{CaSiO_3}}\\
\end{itemize}

Fumée:
\begin{itemize}
\item{\chemical{CO} et \chemical{CO_2} en nombre de mole égale}\\
\end{itemize}

\textbf{Réactions chimiques ayants lieu dans le haut-fourneau:}\\

\underline{Formation de \chemical{CO_2} et \chemical{CO} à partir du coke et de l'air injecté}

\begin{center}
\[
\chemical{C}\chemical{+}\chemical{O_2}
\chemical{->}{oxydation}{}
\chemical{CO_2}
\]

\[
\chemical{CO_2}\chemical{+}\chemical{C}
\chemical{->}{réaction}{\text{(en partie)}}
\chemical{CO}
\]
\end{center}

\underline{Formation de la fonte}
\begin{center}
\[
\chemical{Fe_2O_3}\chemical{+}\chemical{3CO}
\chemical{->}
\chemical{2Fe}\chemical{+}\chemical{3CO_2}
\]

\[
\chemical{Fe_2O_3}\chemical{+}\chemical{CO}
\chemical{->}
\chemical{2FeO}\chemical{+}\chemical{CO_2}
\]

\[
\chemical{FeO}\chemical{+}\chemical{CO}
\chemical{->}
\chemical{Fe}\chemical{+}\chemical{CO_2}
\]

La fonte et constitué de 90 à 95\% de fer ainsi que de carbone (provenant de la coke) de silicium provenant de la réduction du \chemical{SiO_2}

\[
\chemical{SiO_2}\chemical{+}\chemical{2CO}
\chemical{->}
\chemical{Si}\chemical{+}\chemical{2CO_2}
\]
\end{center}

\underline{Formation du laitier}
\begin{center}
\[
\chemical{CaCO_3}
\chemical{->}
\chemical{CaO}\chemical{+}\chemical{CO_2}
\]

\[
\chemical{CaO}\chemical{+}\chemical{SiO_2}
\chemical{->}
\chemical{CaSiO_3}
\]

\[
\chemical{CaO}\chemical{+}\chemical{Al_2O_3}
\chemical{->}
\chemical{CaAl_2O_4}
\]

Une partie du fer (\chemical{FeO}) et perdu dans le laitier en se combinant à de la silice

\[
\chemical{FeO}\chemical{+}\chemical{SiO_2}
\chemical{->}
\chemical{FeSiO_3}
\]
\end{center}




\end{document}
