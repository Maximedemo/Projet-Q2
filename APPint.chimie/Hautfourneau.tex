\documentclass[11pt,a4paper]{report}
\usepackage[utf8]{inputenc}
\usepackage[francais]{babel}
\usepackage[T1]{fontenc}
\usepackage{amsmath}
\usepackage{amsfonts}
\usepackage{graphicx}
\usepackage{amssymb}
\usepackage{fullpage}
\usepackage{fancybox}
\usepackage[Lenny]{fncychap}
\usepackage{gensymb}
\usepackage{color}
\usepackage{array}
\usepackage{url}
\usepackage{here}
\usepackage[final]{pdfpages}
\usepackage{wrapfig}
\usepackage{etex}%chimie
\usepackage{m-pictex,m-ch-en}%chimie
\author{Groupe 118.5}
\title{APP: Haut-fourneau}
\date{\today}
%\thispagestyle{empty}
%\setcounter{page}{0}
\begin{document}

\maketitle

\textbf{Quantités de départ:}\\

1 tonne de minerai de fer composé de:
\begin{itemize}
\item{38\% de \chemical{Fe_2O_3}}
\item{14\% de \chemical{FeO}}
\item{24\% de \chemical{Si}}
\item{10\% de \chemical{CaCO_3}}
\item{4\% de \chemical{Al_2O_3}}
\item{10\% de \chemical{H_2O}}\\
\end{itemize}


Une certaine quantité de coke (à déterminer):
\begin{itemize}
\item{90\% de \chemical{C}}
\item{10\% d'impuretés qui n'entre pas en compte dans nos équations}
\end{itemize}
On sait que 52,5\% de masse de cette coke va être utilisé exclusivement à fournir de la chaleur.\\

\textbf{Quantités à l'arrivé:}\\

Fonte:
\begin{itemize}
\item{92,9\% de \chemical{Fe}}
\item{6,1\% de \chemical{C}}
\item{1\% de \chemical{Si}}\\
\end{itemize}

Laitier:
\begin{itemize}
\item{0,6\% de \chemical{FeSiO_3}}
\item{99,4\% de \chemical{CaAl_2O_4}, \chemical{CaSiO_3}}\\
\end{itemize}

Fumée:
\begin{itemize}
\item{\chemical{CO} et \chemical{CO_2} en nombre de mole égale}
\end{itemize}

\end{document}
